\documentclass[]{beamer}

\usepackage[polish]{babel}
\usepackage[OT4]{fontenc}
\usepackage[utf8]{inputenc}

\usepackage{lipsum}
\usepackage{multimedia}
\usepackage{graphics}
\usepackage{verbatim}
\usepackage{listings}
\usepackage{tabularx}
\usepackage{amsmath}
\usepackage{url}

\definecolor{mygreen}{rgb}{0,0.6,0}
\definecolor{mygray}{rgb}{0.5,0.5,0.5}
\definecolor{mymauve}{rgb}{0.58,0,0.82}

\lstset{
  % choose the background color; you must add \usepackage{color} 
  % or \usepackage{xcolor}
  backgroundcolor=\color{white},
  % the size of the fonts that are used for the code
  basicstyle=\tiny,        
  % sets if automatic breaks should only happen at whitespace
  breakatwhitespace=false,        
   % sets automatic line breaking
  breaklines=true,                
  % sets the caption-position to bottom
  captionpos=none,                 
  % comment style   
  commentstyle=\color{mygreen},    
  % if you want to delete keywords from the given language
  deletekeywords={...},            
  % if you want to add LaTeX within your code
  escapeinside={\%*}{*)},          
  % lets you use non-ASCII characters;
  % for 8-bits encodings only, does not work with UTF-8
  extendedchars=false,           
  % adds a frame around the code
  frame=none,                    
  % keeps spaces in text, useful for keeping indentation of code 
  % (possibly needs columns=flexible)
  keepspaces=true,                 
  % keyword style
  keywordstyle=\color{blue},    
  % the language of the code   
  language=C++,                 
  % if you want to add more keywords to the set
  morekeywords={*,...},            
  % where to put the line-numbers; possible values are (none, left, right)
  numbers=none,                    
  % how far the line-numbers are from the code
  numbersep=5pt,                   
  % the style that is used for the line-numbers
  numberstyle=\tiny\color{mygray},
  % if not set, the frame-color may be changed on line-breaks 
  % within not-black text (e.g. comments (green here)) 
  rulecolor=\color{black},        
  % show spaces everywhere adding particular underscores;
  % it overrides 'showstringspaces'
  showspaces=false,                
  % underline spaces within strings only
  showstringspaces=false,          
  % show tabs within strings adding particular underscores
  showtabs=false,                  
  % the step between two line-numbers. If it's 1, each line will be numbered
  stepnumber=2,                    
  % string literal style
  stringstyle=\color{mymauve},     
  % sets default tabsize to 2 spaces
  tabsize=2,                       
  % show the filename of files included with \lstinputlisting; 
  % also try caption instead of title
  title=\lstname                   
}


\author{
  Radosław Grymin \and
  Paweł Stefański
}
\title{Sterowanik rozmyty dla odwróconego wahadła --- propozycja rozwiązania za pomocą logiki rozmytej}
\subtitle{
  Algorytmy Ewolucyjne i Rozmyte\\
  (seminarium)\\
  spec. Technologie Informacyjne w Systemach Automatyki
}
\date{Wrocław 2013}
\institute{
  Wydział Elektroniki\\
  Politechnika Wrocławska
}
\subject{}
\keywords{}

\titlegraphic{}

\input{Beamer-Settings}

\usetheme[]{Warsaw}



\begin{document}
% \usebackgroundtemplate{\includegraphics[width=\paperwidth]{img/bg.jpg}}
\frame{\titlepage}

% Przebieg prezentacji
\begin{frame}
  \frametitle{Przebieg prezentacji}
  \tableofcontents
\end{frame}

% The following causes the table of contents to be shown 
% at the beginning of every subsection. Delete this, if you do not want it.
\AtBeginSubsection[]{
  \frame<beamer>{
    \frametitle{Przebieg prezentacji}
    \tableofcontents[currentsection,currentsubsection]
  }
}


%\section{Klasa regulatorów rozmytych SIRM}
%\subsection{Wprowadzenie}
%\begin{frame}[t, allowframebreaks]{SIRM --- single input rule module}
%	\footnotesize
%	\begin{description}
%		\item[SIRM] --- dynamicznie połączony model regulatora rozmytego o $n$ wejściach i $1$ wyjściu
%	\end{description}
	
%	Dla zwyczajnego regulatora rozmytego, wraz z liniowym wzrostem wejść, eksponencjalnie
%	rośnie złożoność reguł co prowadzi do problemów związanych z formułowaniem reguł.
	
%	\framebreak
	
%	W celu rozwiązania tego problemu dynamiczny interfejs SIRM konfiguruje osobno SIRM
%	dla każdego wejścia w sposób następujący:
%	\begin{equation}
%		SIRM-i: \left\lbrace R_i^j: x_i = A_i^j \Rightarrow f_i=C_i^j  \right\rbrace_{j=1}^{m_i}
%\end{equation}	 

%	\begin{description}
%		\item[$SIRM-i$] --- $SIRM$ dla $i$-tego wejścia
%		\item[$R_i^j$]  --- $j$-ta reguła dla $SIRM-i$
%		\item[$x_i$]    --- $i$-ta wartość wejścia
%		\item[$f_i$]    --- $i$-ta zmienna decyzyjna wyjścia
%		\item[$A_i^j$]  --- $j$-ta funkcja przynależności dla $i$-tego wejścia
%		\item[$C_i^j$]  --- $j$-ta funkcja przynależności dla $i$-tej zmiennej decyzyjnej
		 
%	\end{description}

%end{frame}
	
	\section{Warianty sterowników rozmytych na bazie PID}
\subsection{Sterownik rozmyty typu P}
\begin{frame}
	\centering 
	\begin{equation}
		U = K_p \cdot e
	\end{equation}
	Reprezentacja symboliczna reguły dla sterownika rozmytego typu P \bigskip
	

	\textbf{JEŻELI e jest $\langle WL \rangle$ TO $u$ jest $\langle WL \rangle$}
\end{frame}

\subsection{Sterownik rozmyty typu PD}
\begin{frame}
	\centering
	
	\begin{equation}
		u=K_P \cdot e + K_D \cdot de
	\end{equation}
	
	Reprezentacja symboliczna reguły dla sterownika rozmytego typu PD \bigskip
	
	\textbf{JEŻELI e jest $\langle WL \rangle$ I $\Delta e$ jest $\langle WL \rangle$,  TO $u$ jest $\langle WL \rangle$}
\end{frame}

\subsection{Sterownik rozmyty typu PI}
\begin{frame}
	\centering

	\begin{equation}
		u=K_p \cdot e +K_i \cdot \int\!\! e\, dt
	\end{equation}
	
	\begin{equation}
		du = K_p \cdot de + K_i \cdot e
	\end{equation}
	
	Reprezentacja symboliczna reguły dla sterownika rozmytego typu PI \bigskip
	
	\textbf{JEŻELI e jest $\langle WL \rangle$ I $\Delta e$ jest $\langle WL \rangle$,  TO $\Delta u$ jest $\langle WL \rangle$}
\end{frame}
	
	
	\subsection{Sterownik rozmyty typu PID}
	\begin{frame}
		\centering
		\begin{equation}
			u=K_p \cdot e + K_d \cdot de + K_i \int\! e\,dt
		\end{equation}
		
		Reprezentacja symboliczna reguły dla sterownika rozmytego typu PID \bigskip
		
			\textbf{JEŻELI e jest $\langle WL \rangle$ I $\Delta e$ jest $\langle WL \rangle$ i $\Sigma e$ jest $\langle WL \rangle$,  TO $u$ jest $\langle WL \rangle$}
	\end{frame}
	
	\section{Projektowanie SR}
		\subsection{Konstruowanie bazy reguł}
		\begin{frame}
		\begin{itemize}
			\item wiedza intuicyjna i doświadczenie
			\item standardowa baza reguł \emph{Mac Vicara-Whelena}
		\end{itemize}
		\end{frame}
		
		\begin{frame}{Baza reguł Mac Vicara-Whelena}
			Dla każdego zbioru rozmytego wprowadzono siedem etykiet
			lingwistycznych: duży dodatni $PL$ (\emph{positive large}),
			średni dodatni $PM$ (\emph{positive medium}), mały dodatni $PS$ (\emph{positive small}),
			zerowy $ZO$ ($zero$), mały ujemny $NS$ (\emph{negative small}), średni ujemny $NM$ 
			(\emph{negative medium}) i duży ujemny $NL$.
		\end{frame}
		
		\begin{frame}{Bazy reguł \emph{Mac Vicara-Whelena} --- SR typu PD}
			\begin{figure}
				\includegraphics[width=0.6\textwidth]{img/pd}
			\end{figure}
			W zależności od wartości $e$ i $\Delta e$ (czy duże, czy małe itp.) stosuje się różne
			bazy reguł \emph{Mac Vicara-Whelena} --- można precyzyjniej sterować prędkością
			dochodzenia do wartości zadanej.
		\end{frame}
		
		\subsection{Dokładność sterowania}
			\begin{frame}
			\begin{itemize}
				\item 				,,Ziarnistość'' regulatora zależy od tego ile
				wartości przyjmują zmienne lingwistyczne.
				\item jeśli wymagane jest lepsze rozwiązanie w pobliżu wartości
				zadanej, można rozważyć większą ilość zmiennych lingwistycznych np.
				PZO (\emph{positive zero}) 
			\end{itemize}
			\end{frame}
			
			\subsection{Sterowanie nieliniowe}
			\begin{frame}
				\begin{itemize}
					\item jeśli układy sterowania są liniowe, regulator można dostroić tylko
					w danym punkcie pracy 
					
					\item zmiana punktu pracy --- strojenie
					
					\item przestrajanie automatyczne --- regulatory adaptacyjne ---
					parametry zmieniają się podczas działania w celu poprawy
					działania regulatora
				\end{itemize}
			\end{frame}
			
			\subsection{Strojenie funkcji przynależności}
				\begin{frame}
					\begin{itemize}
						\item Zmiana czynnika skalującego --- \emph{wartość wejściowa}$\cdot$
						\emph{wartość wyjściowa}
						
						\item Zmiana kształtów zbiorów rozmytych
					\end{itemize}
				\end{frame}
\end{document}
