\documentclass[]{beamer}

\usepackage[polish]{babel}
\usepackage[OT4]{fontenc}
\usepackage[utf8]{inputenc}

\usepackage{lipsum}
\usepackage{multimedia}
\usepackage{graphics}
\usepackage{verbatim}
\usepackage{listings}
\usepackage{tabularx}
\usepackage{amsmath}
\usepackage{url}

\definecolor{mygreen}{rgb}{0,0.6,0}
\definecolor{mygray}{rgb}{0.5,0.5,0.5}
\definecolor{mymauve}{rgb}{0.58,0,0.82}

\lstset{
  % choose the background color; you must add \usepackage{color} 
  % or \usepackage{xcolor}
  backgroundcolor=\color{white},
  % the size of the fonts that are used for the code
  basicstyle=\tiny,        
  % sets if automatic breaks should only happen at whitespace
  breakatwhitespace=false,        
   % sets automatic line breaking
  breaklines=true,                
  % sets the caption-position to bottom
  captionpos=none,                 
  % comment style   
  commentstyle=\color{mygreen},    
  % if you want to delete keywords from the given language
  deletekeywords={...},            
  % if you want to add LaTeX within your code
  escapeinside={\%*}{*)},          
  % lets you use non-ASCII characters;
  % for 8-bits encodings only, does not work with UTF-8
  extendedchars=false,           
  % adds a frame around the code
  frame=none,                    
  % keeps spaces in text, useful for keeping indentation of code 
  % (possibly needs columns=flexible)
  keepspaces=true,                 
  % keyword style
  keywordstyle=\color{blue},    
  % the language of the code   
  language=C++,                 
  % if you want to add more keywords to the set
  morekeywords={*,...},            
  % where to put the line-numbers; possible values are (none, left, right)
  numbers=none,                    
  % how far the line-numbers are from the code
  numbersep=5pt,                   
  % the style that is used for the line-numbers
  numberstyle=\tiny\color{mygray},
  % if not set, the frame-color may be changed on line-breaks 
  % within not-black text (e.g. comments (green here)) 
  rulecolor=\color{black},        
  % show spaces everywhere adding particular underscores;
  % it overrides 'showstringspaces'
  showspaces=false,                
  % underline spaces within strings only
  showstringspaces=false,          
  % show tabs within strings adding particular underscores
  showtabs=false,                  
  % the step between two line-numbers. If it's 1, each line will be numbered
  stepnumber=2,                    
  % string literal style
  stringstyle=\color{mymauve},     
  % sets default tabsize to 2 spaces
  tabsize=2,                       
  % show the filename of files included with \lstinputlisting; 
  % also try caption instead of title
  title=\lstname                   
}


\author{
  Radosław Grymin \and
  Paweł Stefański
}
\title{Sterowanik rozmyty dla odwróconego wahadła --- propozycja rozwiązania za pomocą logiki rozmytej}
\subtitle{
  Algorytmy Ewolucyjne i Rozmyte\\
  (seminarium)\\
  spec. Technologie Informacyjne w Systemach Automatyki
}
\date{Wrocław 2013}
\institute{
  Wydział Elektroniki\\
  Politechnika Wrocławska
}
\subject{}
\keywords{}

\titlegraphic{}

\input{Beamer-Settings}

\usetheme[]{Warsaw}
